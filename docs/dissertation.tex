\documentclass{report}

\usepackage{microtype}
\usepackage{hyperref}
\usepackage{listings}
\usepackage{algpseudocode}
\usepackage{xcolor}

\definecolor{codegreen}{rgb}{0,0.6,0}
\definecolor{codegray}{rgb}{0.5,0.5,0.5}
\definecolor{codepurple}{rgb}{0.58,0,0.82}
\definecolor{backcolour}{rgb}{0.95,0.95,0.92}

\lstdefinestyle{mystyle}{
    backgroundcolor=\color{backcolour},
    commentstyle=\color{codegreen},
    keywordstyle=\color{magenta},
    numberstyle=\tiny\color{codegray},
    stringstyle=\color{codepurple},
    basicstyle=\ttfamily\footnotesize,
    breakatwhitespace=false,
    breaklines=true,
    captionpos=b,
    keepspaces=true,
    numbers=left,
    numbersep=5pt,
    showspaces=false,
    showstringspaces=false,
    showtabs=false,
    tabsize=2
}

\lstset{style=mystyle}

\title{Implementing UNIX with Effects Handlers}
\author{Ramsay Carslaw}

\begin{document}


\maketitle

\tableofcontents

\chapter{Introduction}

\chapter{Background}

\section{Algebraic Effects and Effect Handlers}

Algebraic effects and their corresponding handlers \cite{plotkin2009handlers}
\cite{pretnar2015introduction} are a programming paradigm that when paired
together offers a novel way to compose programs. When programs are written that
are `black boxes', that is to say their outputs are defined entirely by their
inputs and all functions are pure computation \cite{hughes1989functional}, it is
safe to make assumptions about the inputs. Assumptions like an age will always
be given as an integer or all strings will not exceed the length allocated for
them. When programs interact with the real world it is no longer safe to make
these assumptions. To handle these `side effects' most modern languages
introduce the concept of an exception like the below example:

\begin{lstlisting}
  firstName: String = input(``Enter your first name: '')
  try {
    print(``Hello, '' + firstName)
  } catch InvalidCharacter {
    error(``Invalid Character in first name!'')
  }
  print(``Unreachable with invalid character'')
\end{lstlisting}

Instead it is possible to define \texttt{InvalidCharacter} as an effect with a
handler.

\begin{lstlisting}
  effect InvalidCharacter: () -> String
\end{lstlisting}

\section{\textsc{UNIX}}

\textsc{Unix} \cite{ritchie1978unix} is an operating system designed and
implemented by Dennis M. Ritchie and Ken Thompson at AT\&T's Bell Labs in 1974.
It provides a file system (directories, file protection etc.), a shell,
processes (pipe, fork etc) and a userspace. Since it's first release it has been
reimplemented for a variety of systems.

\chapter{Methods}

\chapter{Results}

\chapter{Conclusion}

\bibliographystyle{unsrt}
\bibliography{diss}

\end{document}
